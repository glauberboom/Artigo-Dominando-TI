\section{Conceitos Básicos} \label{first:figs}

 Segundo \cite{adm} toda organização existe para atende a um objetivo específico sendo pública ou privada. Existe um alvo de mercado a ser alcançado em que a junção de recursos físicos e humanos contribuem para sua execução. Ainda de acordo com \cite{adm} a composição de uma organização há responsabilidades que são inerentes a níveis hierárquicos. Podemos identifica-las nos seguintes grupos: 

\begin{enumerate}
\item Governação Corporativa - responsável pela tomada de decisão, elaborar metas, autorizar investimentos, fiscalizar atividades, dentre outros;
\item Gestão Administrativa - resposável por intermedia a Governança Corporativa com Operacional, gerenciar projetos e produtos, gerir indicadores, dentre outros; e
\item Operacional - responsável pela execução dos projetos e trabalhos operacionais.
\end{enumerate}

Os gurpos se interagem entre sí, de acordo com \cite{IBGC} e \cite{ImplantandoGTI:2012} a Governança Corporativa é responsável por traçar os objetivos, colher os resultados e auditar-lo. Para se alcançar resultados satifátorios é necessário a otimização dos processos e redução das redundâncias dos trabalhos, no qual se torna necessário a utilização da TI. A TI como disciplina precisa ser integrada a Governança Corporotativa como ferramenta meio no alcance dos resultados, que surge então a Governança de TI - GTI.
 
Para falarmos de Governança de TI, iremos abordar primeiramente a Governança Corporativa e sua principais características.

\subsection{Governança Corporativa}
Segundo \cite{IBGC} Governança Corporativa é:

\begin{quote}
	``Sistema pela qual as sociedades são dirigidas, monitoradas e incentivadas, envolvendo o relacionamento entre proprietários, Conselho de Administração, Diretoria e órgãos de controle interno. As boas práticas de governança corporativa convertem princípios em recomendações objetivas alinhando interesses com a finalidade de preservar e otimizar o valor da organização, facilitando seu acesso ao capital e contribuindo para a sua longevidade." \cite[p.17]{IBGC}
\end{quote}


De acordo com o \cite{IBGC} os princípios de Governança Corporativa são, transparência, equidade, prestação de contas e responsabilidade corporativa.

\textbf{Transparência} trata de diceminar a informação de resultados e ações. 

\textbf{Equidade} destaca tratamento igualitário a todos os acionistas. 

\textbf{Prestação de contas} trata da responsabilização de seus atos e omissões. 

\textbf{Responsabilidades corporativas} refere que os agentes de governança, devem zelar pela sustentabilidade das organizações contribuindo por sua longevidade.

A governança corporativa é composta pelo conselho de administração, conselho fiscal e auditoria independente. Todos fazem parte da tomada de decisão sobre os investimentos e projetos que alcançam seus objetivos. Nos órgãos federais a composição da governança corporativa são: o povo, os órgãos auditores e a administração de cada órgão.

	
\subsection{Governança de TI - GTI} \label{sec:govti}

A Governança de TI - GTI, baseado na \cite{GTI:TCU} e \cite{ImplantandoGTI:2012}, é o sistema pelo qual o uso atual e futuro da TI são dirigidos e controlados pela Governança Corporativa para garantir que a TI sustente, monitore, e estenda as estratégias e objetivos de negócio da organização. 

De acordo com \cite{ImplantandoGTI:2012}, são competências da Governança de TI: 

\begin{enumerate}
\item Prover o alinhamento estratégico da TI ao negócio;
\item Prover ferramentas que garantam a continuidade do negócio contra interrupções e falhas;
\item Prover junto a áreas de controle o alinhamento a regulamentos e normas internas e externas ao órgão.

\end{enumerate}

Os objetivos da GTI são inerentes ao negócio, o principal segundo \cite{ImplantandoGTI:2012}, é alinhar a TI aos requisitos do negócio, que inclui, soluções de apoio a tomada de decisões e apoio aos processos do negócio. O último proporciona garantia da continuidade dos serviços e a minimização da exposição do negócio aos riscos de TI. 

Ao aplicar a Governança de TI encontra-se dentro do principal objetivo, vários outros relacionados a GTI, que atendem as necessidades da organização. São eles:  

\begin{enumerate}
\item Traduzir as estratégias de negócio em plano para sistemas, aplicações, soluções, processos e infraestrutura, desenvolvimento de competências, estratégias de sourcing e de segurança da informação, dentre outras;
\item Prover o alinhamento e priorização da iniciativas de TI com a estratégia do negócio em um portifólio de TI que liga a estratégia e as ações do dia a dia;
\item Alinha a arquitetura de TI planejar os projetos e priorizá-los no presente e no futuro;
\item Prover a implantação de melhoria nos processos operacional e de gestão necessários para atender aos serviços de TI.
\end{enumerate}

Essas atividades levam a TI tomar um rumo alinhado as necessidade da organização, sem desperdiçar o potencial que a TI proporciona e assim adicionando valor ao negócio.

Segundo \cite{GTI:TCU}, a Governança de TI tem foco no direcionamento e monitoramento das práticas de gestão e uso da TI de uma organização, tendo como indutor e principal beneficiário a alta administração da instituição.

O objetivo de negócio da organização deve ser identificado e repassado a TI, para que possa criar estratégias e técnicas, utilizando  metodologias disponíveis no mercado para alcançá-los. 

A governança de TI como disciplina não segue solitária na busca da valorização da TI. Existem vários modelos de referência que abordam a TI em sua totalidade, como:

 \begin{itemize}
 
 \item ISO/IEC 38500 - Trata de fornecer uma estrutura de princípios para os dirigentes utilizarem na avaliação, no gerenciamento e no monitoramento do uso da TI em suas organizações. 
 \item CobiT - Trata em contribuir para o sucesso da entrega de produtos e serviços de TI a partir da perspectiva das necessidades do negócio, com um foco mais acentuado no controle que na execução.
 \item Val IT - Trata de um modelo que reforça a necessidade de demonstração do retorno que a TI fornece ao negócio.
 \item Risck IT - O modelo é dedicado a auxiliar no gerenciamento de riscos relacionados a TI.
  
 \end{itemize}
 

\subsection{Governança de TI na Administração Pública}

Segundo \cite{ImplantandoGTI:2012} e \cite{SISP} , A partir da década de 90 o governo federal identificou a necessidade de organizar e investir na implementação de GTI em seus órgãos. Foi então que criou o Sistema de Administração de Recursos de Informação e Informática (SISP), com o objetivo de organizar a operação, o controle, a supervisão e coordenação dos recursos de informática da Administração direta, autárquica e fundacional do poder executivo Federal. 

O SISP é responsável por direcionar os órgão federais a alcançar alinhamento estratégico. Já aspectos referentes a segurança da informação no Governo Federal é de responsabilidade do Gabinete de Segurança Institucional da  Presidência da República, através do Departamento de Segurança da Informação e Comunicações. A fiscalização da tecnologia da informação da Administração Pública Federal e de responsabilidade do Tribunal de Contas da União - TCU.

Essas instituições auxiliam os órgãos federais a promoverem o poder alcançado pela valorização da TI na Administração, pois a administração necessita fornecer serviços com qualidade e segurança a sociedade. 

\subsection{Índice de Governança de TI - iGovTI}\label{subsec:iGovTI}

Segundo \cite{acordaoTCU}, o TCU com o objetivo de avaliar a situação da Governança de TI na Administração Pública Federal, tem realizado levantamento baseado em questionários que abordam práticas de governança e de gestão de TI previstas em leis, regulamentos, normas técnicas e modelos internacionais de boas práticas. 

Esse levantamento é feito desde 2007 e tem o objetivo de acompanhar e manter as informações atualizadas com a situação da governança de tecnologia da informação na Administração Pública Federal - APF.
 
De acordo com \cite{acordaoTCU} o Índice de Governança de TI - iGovTI foi criado em 2010 pelo TCU com o propósito de orientar as organizações públicas no esforço de melhoria da governança e da gestão de TI. O índice tambem permite a avaliação, de um modo geral, a efetividade das ações adotadas para induzir a melhoria da situação de governança de TI na APF.   

Por meio de questionário e a consolidação das respostas recebemos como resultado o iGovTI. As organizações que respondem ao questionário são classificadas e seguimentadas de acordo com perfil, para facilitar a identificação da situação no índice.