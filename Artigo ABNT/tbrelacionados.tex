\section{Trabalhos Relacionados} \label{ses:rela}

A Governança de TI no âmbito federal, é caracterizada por ser uma busca constante pelos órgãos, haja visto que a TI esta em constante inovação.

Foi realizado uma pesquisa em alguns órgãos do Governo Federal da visão de cada um sobre a TI. 

\subsection{Tribunal de Contas da União}

Segundo o \cite{GTI:TCU}, o TCU entende que por ser uma instituição que depende de informação para a realização de seus trabalhos e cada vez mais da TI para adequadamente tratar, analisar, fazer uso, disseminar e proteger essas informações. Além disso, é cada vez maior a automação de processos de trabalho do Tribunal, como meio de se assegurar o alcance e a manutenção de padrões de desempenho e qualidade compatíveis com as necessidades da sociedade brasileira.

Como visto o órgão valoriza a informação, pois os frutos de seus trabalhos dependem do tratamento dessas informações. Quem faz o tratamento e a manipulação é a própria TI, atravéz dos diversos sistemas que foram criados a partir de otimização dos processos.

A Governança de TI do TCU está justamente atrelado a esse objetivo de gerir a informação, que é alem da otimização dos processos, é sim usar os benefício que a informação apresenta pela tecnologia da informação para adicionar valor ao negócio.  

\subsection{Caixa Economica Federal - CEF}

Por se tratar de uma instituição financeira nacional, a CEF precisa suportar as atividades seguras e competitiva para se manter no mercado. Pelo fato de trabalhar com alto capital e varios concorrentes do mercado financeiro.

Com esses atributos a instituição tem segundo \cite{Caixa}, objetivos de Governança de TI bem definidos de eficiência na entrega de soluções de TI, ser inovador nas soluções em TI e ter disponibilidade e performance nos serviços de TI. Esses que contribuem com os objetivos de negócio de liderar acesso a serviços financeiros, maximizar eficiência operacional e de ser o principal agente financeiro do desenvolvimento sustentável brasileiro.

A CEF é uma instituição madura na Governança de TI pois identifica os benefícios da TI desde 1996, época em que começou a investir em TI alinhando as sua estratégias em projetos de TI.  
 
 \subsection{Banco Central do Brasil - Bacen}
 
 Banco central responsável por fiscalizar as instituições financeiras do país é uma referência para os demais órgãos federais na gestão da Governança de TI. 
 
 Segundo \cite{Bacen} antes da implantação de GTI o órgão passava por situações indesejadas na execução das atividades de TI. Tais com projetos com elevados prazos e entrega, atrasos na entrega de projetos, baixa percepção da qualidade dos serviços de TI e baixa precisão dos esforços realizados. Um cenário crítico para uma instituição com grandes responsabilidades. 
 
 Então focaram em otimizar esses processo na adoção de boas práticas para gerência de projetos e serviços. 
 
 Para a gerência de projetos foi adotado como referência o PMBok, com o objetivo de otimização processos de replanejamento, acompanhamento e definição de indicadores de desempenho.
 
 Para a gerência de serviços foi adotado como referência o, Information Technology Infrastructure Library -ITIL. Que ajudou nos processos de gestão de mudanças e incidentes e na definição de indicadores de disponibilidade.
 
 Esses atributos são evidências da implantação de Governança de TI no Bacen, pois os serviços e projetos que o órgão trabalha são voltados a facilitação e otimização das diretrizes e tarefas estabelecidas a ele. Além do alinhamento da TI ao negócio, as ferramentas auxiliam a tomada de decisão através do portifólio de projetos.
 
 