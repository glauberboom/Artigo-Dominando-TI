\section{Conclusão}\label{concl}

Este trabalho foi desenvolvido com um objetivo claro de aprimorar os serviços da TI implementando a Governança de TI na Agência Espacial Brasileira. 

Foi compreendido no decorrer do trabalho que essa tarefa não é simples, é trabalhosa e demanda dedicação por parte dos envolvidos. Não será da noite para o dia que se conseguira otimizar os resultados alcançados pela TI na AEB, mas com a sensibilização da alta direção conquistará a iniciação do programa.

A implantação da Governança de TI demanda de uma mudança cultural na organização para sua completa execução, pois com maus hábitos as boas práticas da GTI são anuladas.

Empreende-se que o governo federal se preocupa com a TI, mas há grandes barreiras que devem ser vencidas como: burocracia, modernização, inovação, cultura organizacional, dentre outras importantes para sua aplicação. 

Os órgãos devem rever os organogramas, haja visto que muitos órgãos foram criados antes das modernizações da Tecnologia e não se adequaram para usufruir de seus benefícios. 

A TI sem uma governança é uma área sem norte. Trabalha apenas para resolver problemas imprecisos de atividades imediatas.

Portanto a TI no governo federal sem estratégia não tem seu valor, pois não é alinhada ao objetivo da instituição que a apadrinha, consequentemente não traz benefício algum para a sociedade.

\section{Trabalhos Futuros}\label{trabFut}

No desenvolvimento do trabalho foi identificado oportunidades que poderão ser abordados em trabalhos futuros.

São eles:

Elaborar o Plano de Mudança da Cultura Organizacional aplicado a AEB, pois neste trabalho abordou-se apenas a sua importância. Tendo em vista que é requisito para sucesso da implementação da Governança de TI.

Desenvolver o Plano de Programa de Governança de TI, contendo a descrição da execução dos projetos, da gerência do custeio entre outras.

Elaborar o Plano de Gerenciamento dos Projetos, contendo o um plano individual para projetos baseando-se nas boas práticas do PMBOK.

Implantando Segurança da Informação na AEB, contendo o plano para se implementar a segurança de TI.
 