\section{Resultados Esperados}\label{sec:res}

Como já visto, nos dias de hoje a TI assume grandes responsabilidades no auxiliar a processos importantes na execução das tarefas organizacionais, mas não se limita a eles.

Espera-se que a alta administração da AEB identifique os benefícios que podem ser encontrados através da TI nos quais foram aplicados ao cenário encontrado no órgão.

Deseja-se que o roteiro proposto na Seção \ref{sec:proposta} possa auxiliar a implementação da governança de TI na AEB e sirva como guia de ações, sabendo que foi criado baseado em modelos e guias de boas práticas disponíveis no mercado. Haja visto que aborda vários itens que vão do início da implementação, até a revisão do programa e a continuidade das tarefas.

Ao implementar o roteiro espera-se que:

\begin{itemize}
\item Auxilie a Implantação da Governança de TI.
\item Criação da diretoria de TI  responsável pela gerência das áreas citadas na Seção
\ref{subsec:estruturaGTI}, distribuindo as tarefas especificas nas divisões.
\item Aumento da maturidade dos processos de TI, e organizacionais como um todo.
\item Mudança da Cultura Organizacional em relação a TI.
\end{itemize}





  